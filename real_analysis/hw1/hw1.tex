% ***********************************************************
% ******************* PHYSICS HEADER ************************
% ***********************************************************
% Version 2
\documentclass[12pt]{article}
\usepackage{kotex}
\usepackage{amsmath} % AMS Math Package
\usepackage{amsthm} % Theorem Formatting
\usepackage{amssymb}    % Math symbols such as \mathbb
\usepackage{graphicx} % Allows for eps images
\usepackage[dvips,letterpaper,margin=1in,bottom=0.7in]{geometry}
\usepackage{tensor}
\usepackage[T1]{fontenc}
\usepackage[utf8]{inputenc}
 % Sets margins and page size
\usepackage{amsmath}

\renewcommand{\labelenumi}{(\alph{enumi})} % Use letters for enumerate
% \DeclareMathOperator{\Sample}{Sample}
\let\vaccent=\v % rename builtin command \v{} to \vaccent{}
\usepackage{enumerate}
\renewcommand{\v}[1]{\ensuremath{\mathbf{#1}}} % for vectors
\newcommand{\gv}[1]{\ensuremath{\mbox{\boldmath$ #1 $}}} 
% for vectors of Greek letters
\newcommand{\uv}[1]{\ensuremath{\mathbf{\hat{#1}}}} % for unit vector
\newcommand{\abs}[1]{\left| #1 \right|} % for absolute value
\newcommand{\avg}[1]{\left< #1 \right>} % for average
\let\underdot=\d % rename builtin command \d{} to \underdot{}
\renewcommand{\d}[2]{\frac{d #1}{d #2}} % for derivatives
\newcommand{\dd}[2]{\frac{d^2 #1}{d #2^2}} % for double derivatives
\newcommand{\pd}[2]{\frac{\partial #1}{\partial #2}} 
% for partial derivatives
\newcommand{\pdd}[2]{\frac{\partial^2 #1}{\partial #2^2}} 
% for double partial derivatives
\newcommand{\pdc}[3]{\left( \frac{\partial #1}{\partial #2}
 \right)_{#3}} % for thermodynamic partial derivatives
\newcommand{\ket}[1]{\left| #1 \right>} % for Dirac bras
\newcommand{\bra}[1]{\left< #1 \right|} % for Dirac kets
\newcommand{\braket}[2]{\left< #1 \vphantom{#2} \right|
 \left. #2 \vphantom{#1} \right>} % for Dirac brackets
\newcommand{\matrixel}[3]{\left< #1 \vphantom{#2#3} \right|
 #2 \left| #3 \vphantom{#1#2} \right>} % for Dirac matrix elements
\newcommand{\grad}[1]{\gv{\nabla} #1} % for gradient
\let\divsymb=\div % rename builtin command \div to \divsymb
\renewcommand{\div}[1]{\gv{\nabla} \cdot \v{#1}} % for divergence
\newcommand{\curl}[1]{\gv{\nabla} \times \v{#1}} % for curl
\let\baraccent=\= % rename builtin command \= to \baraccent
\renewcommand{\=}[1]{\stackrel{#1}{=}} % for putting numbers above =
\DeclareRobustCommand{\rchi}{{\mathpalette\irchi\relax}}
\newcommand{\irchi}[2]{\raisebox{\depth}{$#1\chi$}} % inner command, used by \rchi

\providecommand{\wave}[1]{\v{\tilde{#1}}}
\providecommand{\fr}{\frac}
\providecommand{\RR}{\mathbb{R}}
\providecommand{\NN}{\mathbb{N}}
\providecommand{\MM}{\mathfrak{M}}
\providecommand{\seq}{\subseteq}
\providecommand{\e}{\epsilon}

\newtheorem{prop}{Proposition}
\newtheorem{thm}{Theorem}[section]
\newtheorem{axiom}{Axiom}[section]
\newtheorem{p}{Problem}%[section]
\usepackage{cancel}
\newtheorem*{lem}{Lemma}
\theoremstyle{definition}
\newtheorem*{dfn}{Definition}
 \newenvironment{s}{%\small%
        \begin{trivlist} \item \textbf{Solution}. }{%
            \hspace*{\fill} $\blacksquare$\end{trivlist}}%
% ***********************************************************
% ********************** END HEADER *************************
% ***********************************************************

\begin{document}

{\noindent\Huge\bf  \\[0.5\baselineskip] {\fontfamily{cmr}\selectfont  Assignment \#1}         }\\[2\baselineskip] % Title
{ {\bf \fontfamily{cmr}\selectfont real analysis}\\ {\textit{\fontfamily{cmr}\selectfont     April 4, 2022}}}~~~~~~~~~~~~~~~~~~~~~~~~~~~~~~~~~~~~~~~~~~~~~~~~~~~~~~~~~~~~~~~~~~~~~~~~~~~~~    {\large \textsc{2017-11362 통계학과 박건도}
\\[1.4\baselineskip] 

% problem 1
\begin{p}
Let $(X, \MM, \mu)$ be a measure space, let $Y$ be a non-empty measurable subset of $X$, and let $\MM_Y$ be the collection of subsets of $Y$ belonging to the $\sigma$-algebra $\MM$. Prove that $\MM_Y$ is a $\sigma$-algebra and the measure $\mu$ restricted to $\MM_Y$ is also a measure on $\MM_Y$ (and thus $(Y, \MM_Y, \mu)$ consists a measure space).
\end{p}
\begin{s}
First of all, let's show that $\MM_Y$ is a $\sigma$-algebra.\\
① $Y \subseteq \MM_Y$ is trivial.\\
② $A \in \MM_Y \Rightarrow A^c \in \MM_Y$ \\
\indent Let $A \in \MM_Y$. Then $A \in \MM$ and so does $A^c$.\\
\indent Since $Y \in \MM$, $Y\setminus A = Y \cap A^c \in \MM$.\\
\indent $\therefore Y\setminus A \in \MM_Y$. ($\because Y \cap A^c \subseteq Y$)\\
③ $A_1, A_2, \ldots \in \MM_Y \Rightarrow \bigcup_{k=1}^\infty A_k \in \MM_Y$\\
\indent $A_1, A_2, \ldots \in \MM_Y$ means $A_k \subseteq Y$ for all $k$, i.e., $\bigcup_{k=1}^\infty A_k \subseteq Y$.\\
\indent Since $\MM$ is a $\sigma$-algebra, $\bigcup_{k=1}^\infty A_k \in \MM$ for $A_1, A_2, \ldots \in \MM$.\\
\indent Therefore,
\begin{align*}A_1, A_2, \ldots \in \MM_Y &\Rightarrow \bigcup_{k=1}^\infty A_k \in \MM\ \rm{and}\ \bigcup_{k=1}^\infty A_k \subseteq Y\\&\Rightarrow \bigcup_{k=1}^\infty A_k \in \MM_Y
\end{align*}
Therefore, $\MM_Y$ is a $\sigma$-algebra.\\
Then we have to show that the measure $\mu$ restricted to $\MM_Y$ is a measure on $\MM_Y$. Since $\mu(\varnothing)=0<\infty$, countable additivity remains only to prove.\\
Let $A_1, A_2, \ldots \in \MM_Y$ be disjoint sets. Since $\MM_Y$ belonging to $\MM$, $A_1, A_2, \ldots \in \MM$ and $\mu(\bigcup A_k) = \sum\mu(A_k)$.\\
Therefore, $(Y, \MM_Y, \mu)$ is a measure space.
\end{s}
\bigskip
%problem 2
\begin{p}
Let $(X, \MM, \mu)$ be a measure space such that $\mu(X) < \infty$. Define
\begin{align}
\MM_1 = \left\{ A \in \MM : \mu(A) = 0 \ \rm{or}\ \mu(A) = \mu(X)\right\} .
\end{align}
\v{(1)} Prove that $(X, \MM_1, \mu)$ is a measure space.\\
\v{(2)} Let $f: X \rightarrow \RR$ be a measurable function with respect to the $\sigma$-algebra $\MM$. Prove that $f$ is measurable with respect to the $\sigma$-algebra $\MM_1$ if and only if there exists $c \in \RR$ such that $f(x)=c$ a.e. $[\mu]$.
\end{p}
\begin{s}
\v{(1)} To show that $(X, \MM_1, \mu)$ is a measure space, we have to prove following statements:\\
\indent(i) $\MM_1$ is a $\sigma$-algebra.\\
\indent(ii) $\mu$ restricted to $\MM_1$ is a measure on $(X, \MM_1)$\\
First of all, let's show that $\MM_1$ is a $\sigma$-algebra. It is clear that $X \in \MM_1$. For $A \in \MM_1$, $\mu(A)$ is either $0$ or $\mu(X)$. Then, $\mu(A^c)=\mu(X) - \mu(A) =\mu(X)\ \rm{or}\ 0$, so $A^c \in \MM_1$. And for $A_1, A_2, \ldots \in \MM_1$, there are two cases:\\
$\begin{cases}
\mu(A_1)=\mu(A_2) = \cdots = 0\\
{}^\exists k\ \rm{s.t.}\ \mu(A_k)=\mu(X)
\end{cases}$\\
In first case, $\mu(\bigcup A_i) \le \sum\mu(A_i)=0$ means $\bigcup A_i \in \MM_1$. We can also lead the same conclusion from the other case since $$\mu(X) = \mu(A_k) \le \mu(\bigcup A_i) \le \mu(X).$$
Therefore, $\MM_1$ is a $\sigma$-algebra.
Next, to show that $\mu$ is a measure on $(X, \MM_1)$, we have to show the countable additivity $(\mu(\varnothing)=0<\infty$ is easy to show). Likewise the \v{Solution} of \v{Problem\ 1.}, it is clear that $(X, \MM_1, \mu)$ is a measure space.\\ \\
\v{(2)} To show iff condition, we have to show following 2 statements:\\
\indent(i) $f(x)=c$ a.e. $[\mu]\quad \Rightarrow \quad f$ is measurable on $\MM_1$.\\
${}^\exists N$ s.t. $\mu(N)=0$, $f(x)=c$ where $x\in X\setminus N$.\\
Let's show $\mu(f^{-1}(V))=0$ or $\mu(X)$ for open $V \subseteq \RR$, , so that $f^{-1}(V) \in \MM_1$.\\

If $c \in V$,
\begin{align*}
f^{-1}(V) &= f^{-1}(\{c\}\cup V\setminus\{c\})\\
&= f^{-1}(\{c\}) \cup f^{-1}(V\setminus\{c\})\\
& \supseteq X\setminus N
\end{align*}
$\therefore\mu(X\setminus N) = \mu(X) - \mu(N) = \mu(X)\quad\Rightarrow\quad \mu(f^{-1}(V)) = \mu(X)$\\
If $c \not\in V$, $f(x)=c$ where $x\in X\setminus N$ and it means $f^{-1}(V) \subseteq N$.\\
$\therefore \mu(f^{-1}(V))=0$ ($\because \mu(N)=0$).\\
$\therefore f^{-1}(V) \in \MM_1$.\\

\indent(ii) $f$ is measurable on $\MM_1\quad\Rightarrow\quad f(x)=c$ a.e. $[\mu]$.\\
Let $c = \sup\{a\in\RR: \mu(f^{-1}((a,\infty))=\mu(X)\}$.\\
Since $\lim\limits_{n\to\infty}\mu(f^{-1}((n,\infty))=\mu(\bigcap_{n=1}^{\infty}f^{-1}((n,\infty)))=\mu(f^{-1}(\varnothing))=0$, ${}^\exists N$ s.t. $\mu(f^{-1}(N,\infty))=0$ and $c<N<\infty$.\\
Likewise, ${}^\exists M$ s.t. $\mu(f^{-1}(M,\infty))=\mu(X)$ and $M<c<N$, i.e., $c\in\RR$.\\
 Then for $n\in\NN$, $\mu(f^{-1}((c-\frac{1}{n}, \infty)) = \mu(X)$ and $\mu(f^{-1}((c+\frac{1}{n}, \infty)) = 0$ by the definition of supremum.\\
Then,
\begin{align*}
\mu(X) &= \mu\left(f^{-1}\left(\left(c-\frac{1}{n}, \infty\right)\right)\right) - \mu\left(f^{-1}\left(\left(c+\frac{1}{n}, \infty\right)\right)\right)\\
&= \mu\left(f^{-1}\left(\left(c-\frac{1}{n}, c+\frac{1}{n}\right]\right)\right),\\
\mu(f^{-1}(\{c\})) &=\mu\left(\bigcap_{n=1}^{\infty}f^{-1}\left(\left(c-\frac{1}{n},c+\frac{1}{n}\right]\right)\right)\\
&=\lim_{n\to\infty}\mu\left(f^{-1}\left(\left(c-\frac{1}{n}, c+\frac{1}{n}\right]\right)\right)=\mu(X)
\end{align*}
$\therefore f(x)=c$ a.e. $[\mu]$. ($\because \mu(f^{-1}(\RR\setminus\{c\}))=0$).
\end{s}
\bigskip
% problem 3
\begin{p}
Let $(X, \MM)$ be a measureable space and let $\mu_1, \mu_2,\ldots, \mu_k$ be positive measures on this space. Let $c_1, c_2, \ldots, c_k \ge 0$ be non-negative reals and define $\mu:\MM\rightarrow\RR$ as
\begin{align*}
\mu(E) = \sum_{i=1}^{k} c_i\mu_i(E)\quad\rm{for}\ E\in\MM
\end{align*}
Prove that $\mu$ is a measure. (This measure is denoted by $c_1\mu_1 + \cdots+c_k\mu_k$).
\end{p}
\begin{s}
$\mu(\varnothing)=\sum c_i\mu_i(\varnothing)=0<\infty$.\\
Let $A_1, A_2, \ldots \in \MM$ be disjoint sets. Since each $\mu_i$ is a measure, $\mu_i(\bigcup A_j) = \sum_j\mu_i(A_j)$.
\begin{align*}
\mu\left(\bigcup_{j=1}^\infty A_j\right) &=\sum_{i=1}^{k} c_i \mu_i \left(\bigcup_{j=1}^\infty A_j\right) = \sum_{i=1}^k c_i\sum_{j=1}^\infty \mu_i(A_j)\\
&= \sum_{j=1}^\infty \sum_{i=1}^k c_i\mu_i(A_j)\\
&= \sum_{j=1}^\infty \mu(A_j)
\end{align*}
$\therefore \mu$ is a measure.
\end{s}
\bigskip
% problem 4
\begin{p}
Suppose that $(\RR, \tau)$ is the standard topological space, i.e.,
\begin{align*}
\tau = \left\{ A: A\rm{\ is\ a\ countable\ union\ of\ open\ intervals\ in\ } \RR \right\}.
\end{align*}
Denote by $\mathfrak{B}$ the Borel $\sigma$-algebra associated to this topological space. Prove that the following set belongs to $\mathfrak{B}$.
\begin{align*}
A=\left\{\right. x\in\RR:\ &x=q_1 \sqrt{n_1} + \cdots + q_k \sqrt{n_k}\ \mathrm{for\ some}\ k\in\NN,\ q_1,\ldots,q_k\in\mathbb{Q},\\
&\mathrm{and}\ n_1,\ldots, n_k\in\NN\left.\right\}
\end{align*}
\end{p}
\begin{s}
For $a \in \RR$, $[a, \infty) = \bigcup\limits_{n=1}^\infty \left(a-{1\over n}, \infty\right)$ and $(-\infty, a] = \bigcup\limits_{n=1}^\infty \left(-\infty, a+{1\over n}\right)$, so $[a,\infty), (-\infty,a] \in \tau$ and therefore $\{a\}\in\mathfrak{B}$. Let
\begin{align*}
A_k = \left\{\right. x\in\RR:\ &x=q_1\sqrt{n_1}+\cdots+q_k\sqrt{n_k} \mathrm{\ for\ some}\ q_i\in\mathbb{Q},\ n_i\in\NN,\\
& i=1,\cdots,k\left.\right\}
\end{align*}
Since there exists an injection s.t. $A_k \to \mathbb{Q}^k \times \NN^k$ and $\mathbb{Q}$, $\NN$ are countable, $A_k$ is at most countable and $A_k = \bigcup_{x\in A_k} \{x\} \in \mathfrak{B}.$\\
$\therefore A = \bigcup\limits_{k=1}^{\infty}A_k \in\mathfrak{B}.$
\end{s}
\bigskip
% problem 5
\begin{p}
Let $(X, \MM, \mu)$ be a measure space such that $\mu(X)<\infty$ and let $f$, $g$, $h\in L^1(\mu)$. For $n\in\NN$, define
\begin{align*}
B_n = \left\{x\in X :\ |f(x)| + |g(x)| \le n \right\}\in\MM.
\end{align*}
Prove that
\begin{align*}
\lim_{n\to\infty}\int_{B_n} h\ d\mu = \int_X h\ d\mu .
\end{align*}
\end{p}
\begin{s}
It is clear that $|h|, (|h|\rchi_{B_n})_{n=1}^{\infty}$ are measurable functions.\\
Since $B_n \nearrow X$, $|h|\rchi_{B_n}\nearrow |h|$.\\
Also, $|h|\in L^1(\mu)$ and $|h|\rchi_{B_n} \le |h|$, ${}^\forall n\in\NN$.\\
DCT $\Rightarrow$ $\int_X \left||h|-|h|\rchi_{B_n}\right|\ d\mu \to 0$ as $n\to\infty$.
\begin{align*}
0 \le \left| \int_X h-h\rchi_{B_n}\ d\mu \right| &\le \int_X \left|h-h\rchi_{B_n}\right|\ d\mu\\
&=\int_X |h|(1-\rchi_{B_n})\ d\mu\\
&= \int_X \left||h|-|h|\rchi_{B_n}\right|\ d\mu\to 0
\end{align*}
\begin{align*}
\therefore \int_X h\ d\mu = \lim_{n\to\infty}\int_X h\rchi_{B_n}\ d\mu = \lim_{n\to\infty}\int_{B_n}h\ d\mu.
\end{align*}

\end{s}
\bigskip
% problem 6
\begin{p}
Let $(X,\MM,\mu)$ be a complete measure space and let $(f_n)_{n=1}^\infty$, $(g_n)_{n=1}^\infty$ be sequences of measurable functions such that
\begin{align*}
f_m\le g_n\quad \mathrm{a.e.}\ [\mu]\quad\mathrm{for\ all\ } m,n\in\NN.
\end{align*}
Prove that
\begin{align*}
\sup_{n\in\NN}f_n \le \int_{n\in\NN}g_n\quad\mathrm{a.e.}\ [\mu].
\end{align*}
\end{p}
\begin{s}
First of all, let's show that $f_m \ge g_n$ for all $m, n\in\NN$ a.e. $[\mu].$\\
Let $N_{m,n}=\{x\in X:\ f_m(x) > g_n(x)\}$ so that $\mu(N_{m,n})=0$. Then $0\le\mu(\bigcup_{m,n}N_{m,n})\le\sum_{m,n}\mu(N_{m,n})=0$. Thus, for $N=\bigcup_{m,n}N_{m,n}$, $\mu(N)=0$ and ${}^\exists m,n\in\NN$ s.t. $f_m(x) >g_n(x)$ for all $x\in N$, i.e., $f_m(x) \le g_n(x)$ for ${}^\forall m,n\in\NN,\ {}^\forall x\in X\setminus N$.\\
It suffices to show that $\sup f_n(x) \le \inf g_n(x)$, ${}^\forall x\in X\setminus N$.\\
Let $x\in X\setminus N$ is given and assume that $\sup f_n(x) > \inf g_n(x)$.\\
Then, ${}^\exists M\in\NN$ satisfying followings:
\begin{align*}
\begin{cases}
f_M(x) > \sup f_n(x) - {1\over 2} \left(\sup f_n(x) - \inf g_n(x)\right) = {1\over 2} \left(\sup f_n(x) + \inf g_n(x)\right)\\
g_M(x) < \inf g_n(x) + {1\over 2} \left(\sup f_n(x) - \inf g_n(x)\right) = {1\over 2} \left(\sup f_n(x) + \inf g_n(x)\right)
\end{cases}
\end{align*}
It concludes that $g_M(x) < f_M(x)$, which contradicts to $f_m \le g_n,\ {}^\forall m,n$.\\
$\therefore \sup f_n(x) \le \inf g_n(x)$, ${}^\forall x\in X \setminus N$, $\mu(N) = 0$.\\
$\therefore \sup f_n(x) \le \inf g_n(x)$ a.e. $[\mu]$.
\end{s}
\bigskip
% problem 7
\begin{p}
Let $(X, \MM, \mu)$ be a measure space, and let $f\in L^1(\mu)$. Prove that, for all $\e>0$, there exists a simple measurable function $s$ such that
\begin{align*}
\int_X |f-s|\ d\mu < \e.
\end{align*}
\end{p}
\begin{s}
First of all, let's show above statement holds on $f\ge0$ s.t. $f\in L^1(\mu)$.\\
Thm 1.17 $\Rightarrow$ ${}^\exists$ simple $(s_n)_{n=1}^\infty :\ X \to [0,\infty)$ s.t. $s_n \nearrow f$. Then $s_n \in L^1(\mu)$.\\
Then, we can see following holds:
\begin{align*}
\begin{cases}
(s_n)_{n=1}^\infty, f:\ X \to\RR\ \mathrm{MFs}.\\
\lim_{n\to\infty} s_n(x) = f(x),\ {}^\forall x\in X.\\
{}^\exists f\in L^1(\mu)\ \mathrm{s.t.}\ |s_n(x)| \le f(x),\ {}^\forall x\in X,\ {}^\forall n\in\NN.
\end{cases}
\end{align*}
DCT $\Rightarrow$ $\int_X |f - s_n |\ d\mu \to 0$ as $n\to\infty$.\\
$\therefore {}^\exists N\in\NN$ s.t. $\int_X |f-s_N|\ d\mu < \e/2$ and take $s=s_N$.\\
Since $f = f^+ - f^-$, there exists simple functions $s^+$, $s^-$ s.t. $\int_X |f^+-s^+|\ d\mu < \e/2$ and $\int_X |f^--s^-|\ d\mu < \e/2$.\\
Then for $s = s^+ - s^-$,
\begin{align*}
0 \le \int_X|f-s|\ d\mu \le \int_X|f^+ - s^+| + |f^- - s^-|\ d\mu < \e
\end{align*}

\end{s}
\bigskip
% problem 8
\begin{p}
Let $X=\{1, 2, \ldots, 3n\}$ for some positive integer $n\in\NN$, and let
\begin{align*}
\MM = \{ A \subset X:\ |\{3k-2, 3k-1, 3k\}\cap A|=0\ \mathrm{or}\ 3\ \mathrm{for\ all}\ 1\le k \le n\}.
\end{align*}
\v{(1)} Prove that $(X, \MM)$ is a measurable space.\\
\v{(2)} Let $(a_k)_{k=1}^{3n}$ be a sequence of non-negative reals. Define $\mu:\MM\to[0,\infty)$ as
\begin{align*}
\mu(A) = \sum_{i\in A} a_i.
\end{align*}
Prove that $\mu$ is a measure on $(X, \MM)$ and hence $(X, \MM, \mu)$ is a measure space.\\
\v{(3)} Suppose that the sequence $(a_k)_{k=1}^{3n}$ given above satisfies $a_1 = a_2 = a_3 = 0$ and $a_k>0$ for all $k\ge 4$. Prove that the measure space $(X, \MM, \mu)$ is not complete and derive the completion of the measure space $(X, \MM, \mu)$.
\end{p}
\begin{s}
\v{(1)} To show that $(X,\MM)$ is a measurable space, we have to show three things:\\
\indent ① $|\{3k-2,3k-1,3k\}\cap X|=3$ for all $k=1,2,\ldots,n\quad\Rightarrow\quad X\in\MM$\\
\indent ② Assume $A\in\MM$. For $k=1,2,\ldots,n$,
\begin{align*}
\begin{cases}
|\{3k-2,3k-1,3k\}\cap A| = 3 \quad&\Rightarrow\quad \{3k-2,3k-1,3k\}\subseteq A\\
&\Rightarrow\quad |\{3k-2,3k-1,3k\}\cap A^c|=0\\
|\{3k-2,3k-1,3k\}\cap A| = 0 \quad&\Rightarrow\quad |\{3k-2,3k-1,3k\}\cap A^c| = 3
\end{cases}
\end{align*}
\indent $\therefore A^c \in \MM$.\\
\indent ③ $A_1, A_2, \ldots \in \MM$ is given.\\
\indent For $k=1,2,\ldots,n$,\\
\indent\indent i) $|\{3k-2,3k-1,3k\}\cap A_i|=0$, ${}^\forall i$
\begin{align*}
\varnothing &= \bigcup_{i=1}^\infty(\{3k-2,3k-1,3k\}\cap A_i)\\
&=\{3k-2,3k-1,3k\}\cap \left(\bigcup_{i=1}^\infty A_i\right)
\end{align*}
\indent\indent $\therefore\left|\{3k-2,3k-1,3k\}\cap \bigcup A_i\right|=0$\\
\indent\indent ii) ${}^\exists A_i$ s.t. $|\{3k-2,3k-1,3k\}\cap A|=3$
\begin{align*}
\{3k-2,3k-1,3k\}&\subseteq \bigcup_{i=1}^\infty(\{3k-2,3k-1,3k\}\cap A)\\
&=\{3k-2,3k-1,3k\}\cap \bigcup A_i \\
&\subseteq \{3k-2,3k-1,3k\}
\end{align*}
\indent\indent $\therefore\left|\{3k-2,3k-1,3k\}\cap \bigcup A_i\right|=3$\\
\indent $\therefore \bigcup A_i\in\MM$\\
$\therefore (X,\MM)$ is a measurable space.\\ \\
\v{(2)} Since $\mu(\varnothing)=0<\infty$, it suffices to show countable additivity of $\mu$.\\
Let $A_1, A_2, \ldots \in \MM$ be disjoint sets.\\
Since $|X|<\infty$ and $\MM\subseteq 2^X$, $|\MM|<\infty$ and ${}^\exists N$ s.t. $\bigcup_{n=1}^{\infty}A_n = \bigcup_{n=1}^{N}A_n$. Then, when we show that $A_1, A_2\in\MM$ disjoint $\Rightarrow$ $\mu(A_1\cup A_2)=\mu(A_1)+\mu(A_2)$, it can be easily expanded to conclusion by induction.
\begin{align*}
\mu(A_1 \cup A_2) = \sum_{i\in A_1\cup A_2} a_i = \sum_{i\in A_1} a_i + \sum_{i\in A_2} a_i = \mu(A_1) + \mu(A_2).
\end{align*}
$\therefore (X,\MM,\mu)$ is a measure space.\\
\\
\v{(3)} First of all, let's show that given measure space $(X,\MM,\mu)$ is not complete.\\
Let $N = \{1, 2, 3\}\in\MM$ and $A=\{1\}\subseteq N$.\\
Then $\mu(N)=0$ but $A\not\in\MM$ ($\because |\{1,2,3\}\cap A|=1$), so we can conclude that $(X,\MM,\mu)$ is not compelete.\\
The completion of above measure space is,
\begin{align*}
\MM^* =\{E\seq X:\ {}^\exists A, B\in\MM\ \mathrm{s.t.}\ A\seq E \seq B,\ \mu(B\setminus A)=0\}
\end{align*}
Let's find $N\seq X$ s.t. $\mu(N)=0$.
\begin{align*}
\mu(N) = \sum_{i\in N} a_i = 0\quad\Rightarrow\quad N=\varnothing\ \mathrm{or}\ a_i=0,\ {}^\forall i\in N
\end{align*}
$\therefore N\in 2^{\{1,2,3\}}$\\
$\therefore \MM^* = \{N\cup E:\ E\in\MM,\ N\in 2^{\{1,2,3\}}\}$.\\
Extend $\mu$ as $\mu(E)=\mu(E\setminus\{1,2,3\})$ for $E\in\MM^*$.\\
Then, $(X, \MM^*, \mu)$ is a complete measure space.
\end{s}
\end{document}
